% Copyright 2019 by Till Tantau
%
% This file may be distributed and/or modified
%
% 1. under the LaTeX Project Public License and/or
% 2. under the GNU Public License.
%
% See the file doc/generic/pgf/licenses/LICENSE for more details.


%
% This file provides utility commands that are used throughout pgf.
%
% For most commands, the definition of these commands is just given
% below. We cannot use the LaTeX definition of these commands since
% LaTeX may not be the current format and since LaTeX packages tend to
% redefine these commands.
%
% For some commands the actual definition of the format (like latex or
% context) is to be preferred over the generic definition below. In
% this case, the definition of the format is installed when the file
% pgfutil-XXXX.tex is read, where XXXX is the format name (latex,
% plain, or context).



\catcode`\@=11\relax


% Which format is loaded?
\newif\ifpgfutil@format@is@latex
\newif\ifpgfutil@format@is@plain
\newif\ifpgfutil@format@is@context

% Aliases

\let\pgfutil@exp\romannumeral
\chardef\pgfutil@exp@end=0

% Simple stuff

% \pgfutil@trimspaces{<token list>}
%
% Removes spaces on both sides of <token list>.  Same as trimspaces.sty
\catcode`\Q=3
\def\pgfutil@trimspaces#1{%
  \romannumeral-`0\pgfutil@trimspaces@\noexpand#1Q Q}
\def\pgfutil@trimspaces@#1 Q{\pgfutil@trimspaces@@#1Q}
\def\pgfutil@trimspaces@@#1Q#2{#1}
\catcode`\Q=11

% \pgfutil@ifx{<token 1>}{<token 2>}{<true code>}{<false code>}
%
% This macro is expandable.
\long\def\pgfutil@ifx#1#2{%
  \ifx#1#2%
    \expandafter\pgfutil@firstoftwo
  \else
    \expandafter\pgfutil@secondoftwo
  \fi}

% \pgfutil@ifstrequal{<lhs>}{<rhs>}{<true code>}{<false code>}
%
% Check if <lhs> and <rhs> are equal by string comparison
%
% This macro is expandable.
\long\def\pgfutil@ifstrequal#1#2{%
  \ifnum\pgfsys@strcmp{\pgfutil@unexpanded{#1}}{\pgfutil@unexpanded{#2}}=0
    \expandafter\pgfutil@firstoftwo
  \else
    \expandafter\pgfutil@secondoftwo
  \fi
}

% \pgfutil@ifxempty{<token 1>}{<true code>}{<false code>}
%
% Shortcut for \pgfutil@ifx\pgfutil@empty{<token 1>}{<true code>}{<false code>}
%
% This macro is expandable.
\def\pgfutil@ifxempty{\pgfutil@ifx\pgfutil@empty}

% pgfutil@ifempty
%
% Check if #1 is empty. No expansion is performed on #1.
%
% This macro is expandable.
\long\def\pgfutil@ifempty#1{%
  \if\relax\detokenize{#1}\relax
    \expandafter\pgfutil@firstoftwo
  \else
    \expandafter\pgfutil@secondoftwo
  \fi}

% \pgfutil@ifundefined{<macro name with backslash>}
%  {<is undefined code>}{<is defined code>}
%
% This will let #1 to \relax if it undefined.
%
% This macro is expandable.
\def\pgfutil@ifundefined#1{%
  \expandafter\pgfutil@ifx\csname#1\endcsname\relax}

% A variant of \pgfutil@ifundefined which will NOT let #1 to \relax it
% is undefined.
\def\pgfutil@IfUndefined#1{%
  \ifcsname#1\endcsname
    \expandafter\pgfutil@secondoftwo
  \else
    \expandafter\pgfutil@firstoftwo
  \fi
}

\long\def\pgfutil@firstofone#1{#1}
\long\def\pgfutil@firstoftwo#1#2{#1}
\long\def\pgfutil@secondoftwo#1#2{#2}
\def\pgfutil@empty{}
\long\def\pgfutil@gobble@until@relax#1\relax{}
\long\def\pgfutil@gobble#1{}
\long\def\pgfutil@gobbletwo#1#2{}
\def\pgfutil@namedef#1{\expandafter\def\csname #1\endcsname}
\def\pgfutil@namelet#1{\expandafter\pgfutil@@namelet\csname#1\endcsname}
\def\pgfutil@@namelet#1#2{\expandafter\let\expandafter#1\csname#2\endcsname}
\long\def\pgfutil@g@addto@macro#1#2{%
  \begingroup
    \pgfutil@toks@\expandafter{#1#2}%
    \xdef#1{\the\pgfutil@toks@}%
 \endgroup}
\newif\ifpgfutil@tempswa
\newif\ifpgfutil@tempswb

% \pgfutil@pushmacro
% \pgfutil@popmacro
%
% Macro stack.  Allows pushing a macro over \egroup.  Taken from ConTeXt

\let\pgfutil@pushmacro@string\pgfutil@empty

\def\pgfutil@pushmacro#1{%
    \xdef\pgfutil@pushmacro@string{\string#1}%
    \ifcsname pgfutil@pushedmacro@\pgfutil@pushmacro@string\endcsname\else
        % \newcount is \outer in Plain
        \csname newcount\expandafter\endcsname\csname pgfutil@pushedmacro@\pgfutil@pushmacro@string\endcsname
    \fi
    \global\advance\csname pgfutil@pushedmacro@\pgfutil@pushmacro@string\endcsname 1\relax
    \global\expandafter\let\csname\the\csname pgfutil@pushedmacro@\pgfutil@pushmacro@string\endcsname\pgfutil@pushmacro@string\endcsname#1%
}

\def\pgfutil@popmacro#1{%
    \xdef\pgfutil@pushmacro@string{\string#1}%
    \expandafter\let\expandafter#1\csname\the\csname pgfutil@pushedmacro@\pgfutil@pushmacro@string\endcsname\pgfutil@pushmacro@string\endcsname
    \global\advance\csname pgfutil@pushedmacro@\pgfutil@pushmacro@string\endcsname -1\relax
}

% pgfutil@ifnextchar

\long\def\pgfutil@ifnextchar#1#2#3{%
  \let\pgfutil@reserved@d=#1%
  \def\pgfutil@reserved@a{#2}%
  \def\pgfutil@reserved@b{#3}%
  \futurelet\pgfutil@let@token\pgfutil@ifnch}
\def\pgfutil@ifnch{%
  \ifx\pgfutil@let@token\pgfutil@sptoken
    \let\pgfutil@reserved@c\pgfutil@xifnch
  \else
    \ifx\pgfutil@let@token\pgfutil@reserved@d
      \let\pgfutil@reserved@c\pgfutil@reserved@a
    \else
      \let\pgfutil@reserved@c\pgfutil@reserved@b
    \fi
  \fi
  \pgfutil@reserved@c}
{%
  \def\:{\global\let\pgfutil@sptoken= } \:
  \def\:{\pgfutil@xifnch} \expandafter\gdef\: {\futurelet\pgfutil@let@token\pgfutil@ifnch}
}

% pgfutil@ignorespaces

\def\pgfutil@ignorespaces
  {\ifx\pgfutil@let@token\pgfutil@sptoken
     \expandafter\pgfutil@ignorespaces@helper
   \else
     \expandafter\pgfutil@next
   \fi}

{
  \def\:{\pgfutil@ignorespaces@helper}
  \expandafter\gdef\: {\futurelet\pgfutil@let@token\pgfutil@ignorespaces}
}

% pgfutil@in@

\newif\ifpgfutil@in@

% Usage:
% \pgfutil@in@{one}{three two one}
% \ifpgfutil@in@
%     -> will be true!
% \else
% \fi
%
% \pgfutil@in@{,}{1234,456567}
% \ifpgfutil@in@
%     -> will be true!
% \else
% \fi
\def\pgfutil@in@#1#2{%
 \def\pgfutil@in@@##1#1##2##3\pgfutil@in@@{%
  \ifx\pgfutil@in@##2\pgfutil@in@false\else\pgfutil@in@true\fi}%
 \pgfutil@in@@#2#1\pgfutil@in@\pgfutil@in@@}


% pgfutil@for

\def\pgfutil@nnil{\pgfutil@nil}
\def\pgfutil@fornoop#1\@@#2#3{}
\long\def\pgfutil@for#1:=#2\do#3{%
  \expandafter\def\expandafter\pgfutil@fortmp\expandafter{#2}%
  \ifx\pgfutil@fortmp\pgfutil@empty \else
    \expandafter\pgfutil@forloop#2,\pgfutil@nil,\pgfutil@nil\@@#1{#3}\fi}
\long\def\pgfutil@forloop#1,#2,#3\@@#4#5{\def#4{#1}\ifx #4\pgfutil@nnil \else
       #5\def#4{#2}\ifx #4\pgfutil@nnil \else#5\pgfutil@iforloop #3\@@#4{#5}\fi\fi}
\long\def\pgfutil@iforloop#1,#2\@@#3#4{\def#3{#1}\ifx #3\pgfutil@nnil
       \expandafter\pgfutil@fornoop \else
      #4\relax\expandafter\pgfutil@iforloop\fi#2\@@#3{#4}}
\def\pgfutil@tfor#1:={\pgfutil@tf@r#1 }
\long\def\pgfutil@tf@r#1#2\do#3{\def\pgfutil@fortmp{#2}\ifx\pgfutil@fortmp\pgfutil@space\else
    \pgfutil@tforloop#2\pgfutil@nil\pgfutil@nil\@@#1{#3}\fi}
\long\def\pgfutil@tforloop#1#2\@@#3#4{\def#3{#1}\ifx #3\pgfutil@nnil
       \expandafter\pgfutil@fornoop \else
      #4\relax\expandafter\pgfutil@tforloop\fi#2\@@#3{#4}}

\def\pgfutil@space{ }

% pgfutil@IfFileExists

\chardef\pgfutil@inputcheck0
\def\pgfutil@IfFileExists#1#2#3{%
  \openin\pgfutil@inputcheck=#1 %
  \ifeof\pgfutil@inputcheck
     #3\relax
  \else
    #2\relax
  \fi
  \closein\pgfutil@inputcheck}

\def\pgfutil@InputIfFileExists#1#2#3{\pgfutil@IfFileExists{#1}{\input #1\relax#2}{#3}}%


% pgfutil@loop (from plain.tex)

\def\pgfutil@loop#1\pgfutil@repeat{\def\pgfutil@body{#1}\pgfutil@iterate}
\def\pgfutil@iterate{\pgfutil@body \let\pgfutil@next\pgfutil@iterate \else\let\pgfutil@next\relax\fi \pgfutil@next}
\let\pgfutil@repeat=\fi % this makes \loop...\if...\repeat skippable

% \pgfutil@switch
%
% #1: comparison function of the form
%     \comp{<lhs>}{<rhs>}{<true code>}{<false code>}
%     Has to be expandable!
% #2: lhs
% #3: sequence of rhs-value pairs
% #4: default statement in case of match
% #5: default statement in case of no match
%
% Example:
%
% \pgfutil@switch\pgfutil@ifx\pgf@let@token{%
%   {(}{\tikz@movetoabs}%)
%   {+}{\tikz@movetorel}%
%   {-}{\tikz@lineto}%
% }{}{\tikz@expand}

\let\pgfutil@scan@mark\relax
\let\pgfutil@scan@stop\relax

\long\def\pgfutil@switch#1#2#3#4#5{%
  \pgfutil@exp
  \pgfutil@switch@@{#1}%
  {#2}#3% user-defined cases
  {#2}{}% default case
  \pgfutil@scan@mark{#4}% true code
  \pgfutil@scan@mark{#5}% false code
  \pgfutil@scan@stop
}
\long\def\pgfutil@switch@@#1#2#3#4{%
  #1{#2}{#3}%
    {\pgfutil@switch@end{#4}}%
    {\pgfutil@switch@@{#1}{#2}}%
}
\long\def\pgfutil@switch@end#1#2#3\pgfutil@scan@mark#4#5\pgfutil@scan@stop{\pgfutil@exp@end#1#4}

% aux-read-hook

\let\pgfutil@aux@read@hook=\relax


% Tokens for the end of the typesetting -- they will be added at the
% end of every job (hopefully...).

\newtoks\pgfutil@everybye


%
% PDF-Resource management -- might seem strange that this is here, but
% it turns out to be the correct place since latex and context handle
% this stuff quite differently
%

% Adding something to the pdf-resources:

\def\pgfutil@addpdfresource@extgs#1{\pgf@sys@addpdfresource@extgs@plain{#1}}
\def\pgfutil@addpdfresource@colorspaces#1{\pgf@sys@addpdfresource@colorspaces@plain{#1}}
\def\pgfutil@addpdfresource@patterns#1{\pgf@sys@addpdfresource@patterns@plain{#1}}
\def\pgfutil@setuppdfresources{\pgf@sys@setuppdfresources@plain}


% Inserts the argument at the begin of the current page (hopefully)

\let\pgfutil@insertatbegincurrentpage=\relax


% Ragged left and ragged right

\def\pgfutil@raggedright{\rightskip\z@ plus2em \spaceskip.3333em \xspaceskip.5em\relax}
\def\pgfutil@raggedleft{\leftskip\z@ plus2em \rightskip\z@ \spaceskip.3333em \xspaceskip.5em\parfillskip0pt\relax}


% Error Messages
\def\pgfutil@packageerror#1#2#3{\errhelp{#3}\errmessage{Package #1 Error: #2}}
\def\pgfutil@packagewarning#1#2{\immediate\write17{Package #1: Warning! #2.}}
\def\pgferror#1{\pgfutil@packageerror{pgf}{#1}{}}
\def\pgfwarning#1{\pgfutil@packagewarning{pgf}{#1}}


% Library files inclusion


% Include a library file.
%
% #1 = List of names of library file.
%
% Description:
%
% This command includes a list of library files. For each file X in the
% list, the file pgflibraryX.code.tex is included, provided this has
% not been done earlier.
%
% For the convenience of Context users, both round and square brackets
% are possible for the argument.
%
% Example:
%
% \usepgflibrary{arrows}
% \usepgflibrary[patterns,snakes]

\def\usepgflibrary{\pgfutil@ifnextchar[{\use@pgflibrary}{\use@@pgflibrary}}%}
\def\use@pgflibrary[#1]{\use@@pgflibrary{#1}}
\def\use@@pgflibrary#1{%
  \edef\pgf@list{#1}%
  \pgfutil@for\pgf@temp:=\pgf@list\do{%
    \expandafter\pgfkeys@spdef\expandafter\pgf@temp\expandafter{\pgf@temp}%
    \ifx\pgf@temp\pgfutil@empty
    \else
      \expandafter\ifx\csname pgf@library@\pgf@temp @loaded\endcsname\relax%
      \expandafter\let\csname pgf@library@\pgf@temp @loaded\endcsname=\pgfutil@empty%
      \expandafter\edef\csname pgf@library@#1@atcode\endcsname{\the\catcode`\@}
      \expandafter\edef\csname pgf@library@#1@barcode\endcsname{\the\catcode`\|}
      \expandafter\edef\csname pgf@library@#1@dollarcode\endcsname{\the\catcode`\$}
      \catcode`\@=11
      \catcode`\|=12
      \catcode`\$=3
      \pgfutil@InputIfFileExists{pgflibrary\pgf@temp.code.tex}{}{%
          \pgferror{I did not find the pgf library
            '\pgf@temp'. I looked for the file named
            pgflibrary\pgf@temp.code.tex, but could not find it in
            the current texmf trees.}
        }%
      \catcode`\@=\csname pgf@library@#1@atcode\endcsname
      \catcode`\|=\csname pgf@library@#1@barcode\endcsname
      \catcode`\$=\csname pgf@library@#1@dollarcode\endcsname
      \fi%
    \fi
  }%
}


% Include a module file.
%
% #1 = List of names of module files.
%
% Description:
%
% This command includes a list of module files. For each file X in the
% list, the file pgfmoduleX.code.tex is included, provided this has
% not been done earlier.
%
% For the convenience of Context users, both round and square brackets
% are possible for the argument.
%
% Example:
%
% \usepgfmodule{matrix}

\def\usepgfmodule{\pgfutil@ifnextchar[{\use@pgfmodule}{\use@@pgfmodule}}%}
\def\use@pgfmodule[#1]{\use@@pgfmodule{#1}}
\def\use@@pgfmodule#1{%
  \edef\pgf@list{#1}%
  \pgfutil@for\pgf@temp:=\pgf@list\do{%
    \expandafter\ifx\csname pgf@module@\pgf@temp @loaded\endcsname\relax%
      \expandafter\let\csname pgf@module@\pgf@temp @loaded\endcsname=\pgfutil@empty%
      \expandafter\edef\csname pgf@module@#1@atcode\endcsname{\the\catcode`\@}
      \expandafter\edef\csname pgf@module@#1@barcode\endcsname{\the\catcode`\|}
      \expandafter\edef\csname pgf@module@#1@dollarcode\endcsname{\the\catcode`\$}
      \catcode`\@=11
      \catcode`\|=12
      \catcode`\$=3
      \input pgfmodule\pgf@temp.code.tex
      \catcode`\@=\csname pgf@module@#1@atcode\endcsname
      \catcode`\|=\csname pgf@module@#1@barcode\endcsname
      \catcode`\$=\csname pgf@module@#1@dollarcode\endcsname
    \fi%
  }%
}

\def\pgfutilensuremath#1{%
  \ifmmode#1\else$#1$\fi
}


%
% Guess the driver:
%

\begingroup
    \catcode`\"=12
    \edef\pgf@loc@TMPa{"}%
    %
    % prepares file names by checking for double colons.
    %
    % If '#1' is a usual file name without anything fancy, \pgfretval
    % will simply contain it. However, if '#1' contains double colons
    % (introduces, for example, by pdftex because there are white
    % spaces in '#1'), the routine
    % - removes the double colons,
    % - re-inserts them outside of the string.
    %
    % \pgfutilpreparefilename{file.tex} -> file.tex
    % \pgfutilpreparefilename{"A file name".file} -> "A file name.file"
    %
    % The resulting file name is returned in '\pgfretval',
    % furthermore, a *quoted* version of the file name is returned in
    % \pgfretvalquoted. The latter result is to have some sort of
    % output normalisation: if the file name as such contains double
    % quotes, we don't want to insert another set of them.
    %
    % This special handling has the following purposes:
    % 1. Both, treatment of white spaces and double colons is not
    % properly defined in TeX. Thus, we should only work with them if
    % necessary and maintain backwards compatibility as far as
    % possible.
    % 2. It *should* work if there are spaces.
    %
    \gdef\pgfutilpreparefilename#1{%
        \begingroup
            \ifnum\the\catcode`\"=13
                \pgfutilconvertdcolon
            \fi
            \xdef\pgf@temp{#1}%
        \endgroup
        \expandafter\pgfutil@in@\expandafter"\expandafter{\pgf@temp}%
        \ifpgfutil@in@
            \def\pgf@loc@TMPa{\pgfutilstrreplace{"}{}}%
            \expandafter\pgf@loc@TMPa\expandafter{\pgf@temp}%
            \edef\pgfretval{"\pgfretval"}% re-insert quotes! Otherwise, TeX can't use the file name.
            \let\pgfretvalquoted=\pgfretval
        \else
            \let\pgfretval=\pgf@temp
            \edef\pgfretvalquoted{"\pgfretval"}%
        \fi
    }%
    \catcode`\"=13
    \xdef\pgfutilconvertdcolon{%
        \noexpand\def\noexpand"{\pgf@loc@TMPa}%
    }%
\endgroup

% Converts an arbitrary command (without arguments) to a string in which all characters
% have category 12.
%
% #1: a macro name (which takes no arguments)
% #2: a macro name which will be assigned to '#1' converted to string.
%
% This uses '\meaning#1' hackery.
\def\pgfutil@command@to@string#1#2{%
    \expandafter\pgfutil@command@to@string@@\meaning#1\pgfutil@EOI{#2}%
}%
\xdef\pgfutil@glob@TMPa{\meaning\pgfutil@empty}%
\expandafter\def\expandafter\pgfutil@command@to@string@@\pgfutil@glob@TMPa#1\pgfutil@EOI#2{%
    \def#2{#1}%
}%
\begingroup
\catcode`\|=0
\catcode`\\=12
|gdef|pgfutil@backslash@as@other{\}%
|endgroup

% Checks if the token sequence #1 (unexpanded) contains a macro and
% invokes #2 if so and #3 if not.
\def\pgfutilifcontainsmacro#1#2#3{%
    \def\pgf@marshal{#1}%
    \pgfutil@command@to@string\pgf@marshal\pgf@marshal
    \edef\pgf@marshal{\noexpand\pgfutil@in@{\pgfutil@backslash@as@other}{\pgf@marshal}}%
    \pgf@marshal
    \ifpgfutil@in@
        \def\pgf@marshal{#2}%
    \else
        \def\pgf@marshal{#3}%
    \fi
    \pgf@marshal
}%

% Tests if string '#2' starts with pattern '#1'.
%
% If so, it executes #3 and defines \pgfretval to contain the
% suffix which is not equal to #1. Otherwise it executes #4.
\def\pgfutilifstartswith#1#2#3#4{%
    \def\pgfutilifstartswith@ ##1#1##2\pgf@EOI{%
        \def\pgfutil@tmp{##1}%
        \ifx\pgfutil@tmp\pgfutil@empty
            % Ah - a hit!
            %
            % define \pgfretval to be the suffix...
            \def\pgfutil@tmp#1####1\pgf@EOI{%
                \def\pgfretval{####1}%
            }%
            \pgfutil@tmp#2\pgf@EOI%
            %
            % ... and execute the <true> code:
            #3\relax%
        \else
            % hm. No such prefix.
            #4\relax%
        \fi
    }%
    \pgfutilifstartswith@#2--#1\pgf@EOI%
}%

% Usage:
% \pgfutilstrreplace{<token>}{<replacement>}{<string>}
%
% -> will assign the modified string into \pgfretval.
%
% #1: the string to search (one or more tokens)
% #2: zero, one or more tokens which will be inserted instead of '#1'.
% #3: the string to search in
\long\def\pgfutilstrreplace#1#2#3{%
    \def\pgfretval{}%
    \long\def\pgfutil@search@and@replace@@##1#1##2\pgf@EOI{%
        \expandafter\def\expandafter\pgfretval\expandafter{\pgfretval ##1#2}%
        \pgfutil@search@and@replace@loop{#1}{##2}%
    }%
    \pgfutil@search@and@replace@loop{#1}{#3}%
}
\long\def\pgfutil@search@and@replace@loop#1#2{%
    \pgfutil@in@{#1}{#2}%
    \ifpgfutil@in@
        \def\pgf@loc@TMPa{\pgfutil@search@and@replace@@ #2\pgf@EOI}%
    \else
        \expandafter\def\expandafter\pgfretval\expandafter{\pgfretval #2}%
        \let\pgf@loc@TMPa=\relax
    \fi
    \pgf@loc@TMPa
}%

% Solves a linear equation system of size 2x2 using gauss elimination.
%
% It employs TeX register arithmetics to do so.
% #1: should contain 4 sets of braces with matrix entries,
%     {<a11>}{<a12>}
%     {<a21>}{<a22>}
%     where each entry should be a number without unit.
%     It is acceptable if '#1' is a macro which expands to the expected
%     format.
% #2: should contain 2 sets of braces with the right-hand-side,
%     {<r1>}{<r2>}
%     where each entry should be a number without unit.
%     It is acceptable if '#2' is a macro which expands to the expected
%     format.
%
% It will assign \pgfmathresult to contain two sets of braces with the
% result.
%
% Example:
% \pgfutilsolvetwotwoleq{
%     {0.24}{1}
%     {-0.97}{0}
% }{
%     {-7}
%     {18}
% }
% -> yields \pgfmathresult={−18.55618}{−2.54642}
%
% The algorithm employs column pivotisation.
%
% If the matrix is singular, the routine will return {}.
\def\pgfutilsolvetwotwoleq#1#2{%
    \begingroup
        \dimendef\aa=0
        \dimendef\ab=1
        \dimendef\ba=2
        \dimendef\bb=3
        \dimendef\ra=4
        \dimendef\rb=5
        \dimendef\tmpa=6
        \dimendef\tmpb=7
        \edef\pgf@temp{#1}%
        \expandafter\pgfutilsolvetwotwoleq@A\pgf@temp
        \edef\pgf@temp{#2}%
        \expandafter\pgfutilsolvetwotwoleq@r\pgf@temp
        %
        \pgfutilsolvetwotwoleq@ifislarger\aa\ba{%
            % identity "permutation":
            \def\Pa{a}%
            \def\Pb{b}%
        }{%
            % permutation matrix: switch rows!
            \def\Pa{b}%
            \def\Pb{a}%
        }%
        % \pivot := 1/aa
        \tmpa=\csname m\Pa a\endcsname pt %
        \ifdim\tmpa<0pt \tmpa=-\tmpa\fi
        \ifdim\tmpa<0.0001pt
            % singular matrix!
            \let\pgfmathresult=\pgfutil@empty
        \else
            \pgfmathreciprocal@
                {\csname m\Pa a\endcsname}%
            \let\pivot=\pgfmathresult
            %
            % \factor := 1/aa * ba
            \csname \Pb a\endcsname=\pivot\csname \Pb a\endcsname
            \edef\factor{\expandafter\pgf@sys@tonumber\csname \Pb a\endcsname}%
            %
            % bb -= ba/aa * ab
            \tmpa=-\factor\csname \Pa b\endcsname
            \advance\csname \Pb b\endcsname by\tmpa
            %
            % rb -= ba/aa * ra
            \tmpa=-\factor\csname r\Pa\endcsname
            \advance\csname r\Pb\endcsname by\tmpa
            %
            \tmpa=\csname \Pb b\endcsname%
            \ifdim\tmpa<0pt \tmpa=-\tmpa\fi
            \ifdim\tmpa<0.0001pt
                % singular matrix!
                \let\pgfmathresult=\pgfutil@empty
            \else
                % xb := rb / bb (the modified rb and modified bb!)
                \pgfmathdivide@
                    {\expandafter\pgf@sys@tonumber\csname r\Pb\endcsname}
                    {\expandafter\pgf@sys@tonumber\csname \Pb b\endcsname}%
                \expandafter\let\csname pgfmathresult\Pb\endcsname=\pgfmathresult
                %
                % ra := ra - xb * ab
                \tmpa=\csname pgfmathresult\Pb\endcsname\csname \Pa b\endcsname
                \advance\csname r\Pa\endcsname by-\tmpa
                %
                % xa := 1/aa * ra  (the modified ra!)
                \tmpa=\pivot\csname r\Pa\endcsname
                \expandafter\edef\csname pgfmathresult\Pa\endcsname{\pgf@sys@tonumber\tmpa}%
                %
                \edef\pgfmathresult{%
                    {\csname pgfmathresult\Pa\endcsname}%
                    {\csname pgfmathresult\Pb\endcsname}%
                }%
            \fi
        \fi
        \pgfmath@smuggleone\pgfmathresult
    \endgroup
}%

\def\pgfutilsolvetwotwoleq@ifislarger#1#2#3#4{%
    \tmpa=#1
    \ifdim\tmpa<0pt
        \multiply\tmpa by-1
    \fi
    \tmpb=#2
    \ifdim\tmpb<0pt
        \multiply\tmpb by-1
    \fi
    \ifdim\tmpa>\tmpb
        #3%
    \else
        #4%
    \fi
}%

\def\pgfutilsolvetwotwoleqfloat@ifislarger#1#2#3#4{%
    \pgfmathfloatabs@{#1}\let\tmpa=\pgfmathresult
    \pgfmathfloatabs@{#2}\let\tmpb=\pgfmathresult
    \pgfmathfloatlessthan@{\tmpb}{\tmpa}%
    \ifpgfmathfloatcomparison
        #3%
    \else
        #4%
    \fi
}%
\def\pgfutilsolvetwotwoleq@A#1#2#3#4{%
    \def\maa{#1}\def\mab{#2}%
    \def\mba{#3}\def\mbb{#3}%
    \aa=#1pt \ab=#2pt
    \ba=#3pt \bb=#4pt
}
\def\pgfutilsolvetwotwoleq@r#1#2{%
    \ra=#1pt \rb=#2pt
}%
\def\pgfutilsolvetwotwoleqfloat@A#1#2#3#4{%
    \pgfmathfloatparsenumber{#1}\let\maa=\pgfmathresult
    \pgfmathfloatparsenumber{#2}\let\mab=\pgfmathresult
    \pgfmathfloatparsenumber{#3}\let\mba=\pgfmathresult
    \pgfmathfloatparsenumber{#4}\let\mbb=\pgfmathresult
}
\def\pgfutilsolvetwotwoleqfloat@r#1#2{%
    \pgfmathfloatparsenumber{#1}\let\ra=\pgfmathresult
    \pgfmathfloatparsenumber{#2}\let\rb=\pgfmathresult
}%

% Same as \pgfutilsolvetwotwoleq, but using floating point
% arithmetics. The return value is still in fixed point.
\def\pgfutilsolvetwotwoleqfloat#1#2{%
    \begingroup
        \pgfmathfloatcreate{1}{1.0}{-4}% FIXME : use a smaller threshold for FPU?
        \let\thresh=\pgfmathresult
        %
        \edef\pgf@temp{#1}%
        \expandafter\pgfutilsolvetwotwoleqfloat@A\pgf@temp
        \edef\pgf@temp{#2}%
        \expandafter\pgfutilsolvetwotwoleqfloat@r\pgf@temp
        %
        \pgfutilsolvetwotwoleqfloat@ifislarger\maa\mba{%
            % identity "permutation":
            \def\Pa{a}%
            \def\Pb{b}%
        }{%
            % permutation matrix: switch rows!
            \def\Pa{b}%
            \def\Pb{a}%
        }%
        % \pivot := 1/aa
        \expandafter\pgfmathfloatabs@\expandafter{\csname m\Pa a\endcsname}%
        \let\tmpa=\pgfmathresult
        \pgfmathfloatlessthan@{\tmpa}{\thresh}%
        \ifpgfmathfloatcomparison
            % singular matrix!
            \let\pgfmathresult=\pgfutil@empty
        \else
            \expandafter\pgfmathfloatreciprocal@\expandafter{\csname m\Pa a\endcsname}%
            \let\pivot=\pgfmathresult
            %
            % \factor := 1/aa * ba
            \expandafter\pgfmathfloatmultiply@\expandafter{\csname m\Pb a\endcsname}{\pivot}%
            \let\factor=\pgfmathresult
            \expandafter\let\csname m\Pb a\endcsname=\factor
            %
            % bb -= ba/aa * ab
            \expandafter\pgfmathfloatmultiply@\expandafter{\csname m\Pa b\endcsname}{\factor}%
            \let\tmpa=\pgfmathresult
            \expandafter\pgfmathfloatsubtract@\expandafter{\csname m\Pb b\endcsname}{\tmpa}%
            \expandafter\let\csname m\Pb b\endcsname=\pgfmathresult
            %
            % rb -= ba/aa * ra
            \expandafter\pgfmathfloatmultiply@\expandafter{\csname r\Pa\endcsname}{\factor}%
            \let\tmpa=\pgfmathresult
            \expandafter\pgfmathfloatsubtract@\expandafter{\csname r\Pb\endcsname}{\tmpa}%
            \expandafter\let\csname r\Pb\endcsname=\pgfmathresult
            %
            \expandafter\pgfmathfloatabs@\expandafter{\csname m\Pb b\endcsname}%
            \let\tmpa=\pgfmathresult
            \pgfmathfloatlessthan@{\tmpa}{\thresh}%
            \ifpgfmathfloatcomparison
                % singular matrix!
                \let\pgfmathresult=\pgfutil@empty
            \else
                % xb := rb / bb (the modified rb and modified bb!)
                \edef\pgf@marshal{%
                    \noexpand\pgfmathfloatdivide@
                        {\csname r\Pb\endcsname}
                        {\csname m\Pb b\endcsname}%
                }%
                \pgf@marshal
                \expandafter\let\csname pgfmathresult\Pb\endcsname=\pgfmathresult
                \let\tmpa=\pgfmathresult
                %
                % ra := ra - xb * ab
                \expandafter\pgfmathfloatmultiply@\expandafter{\csname m\Pa b\endcsname}{\tmpa}%
                \let\tmpa=\pgfmathresult
                \expandafter\pgfmathfloatsubtract@\expandafter{\csname r\Pa\endcsname}{\tmpa}%
                \expandafter\let\csname r\Pa\endcsname=\pgfmathresult
                %
                % xa := 1/aa * ra  (the modified ra!)
                \expandafter\pgfmathfloatmultiply@\expandafter{\csname r\Pa\endcsname}{\pivot}%
                \expandafter\let\csname pgfmathresult\Pa\endcsname=\pgfmathresult
                %
                \edef\pgfmathresult{%
                    {\csname pgfmathresult\Pa\endcsname}%
                    {\csname pgfmathresult\Pb\endcsname}%
                }%
                \expandafter\pgfutilsolvetwotwoleqfloat@to@pgf@range\pgfmathresult
            \fi
        \fi
        \pgfmath@smuggleone\pgfmathresult
    \endgroup
}%


\def\pgfutilsolvetwotwoleqfloat@to@pgf@range#1#2{%
    \pgfmathfloatcreate{1}{1.6}{4}\let\pgfutilsolvetwotwoleqfloat@to@pgf@range@max=\pgfmathresult
    \pgfmathfloatabs@{#1}%
    \expandafter\pgfmathfloatlessthan@\expandafter{\pgfmathresult}{\pgfutilsolvetwotwoleqfloat@to@pgf@range@max}%
    \ifpgfmathfloatcomparison
        \pgfmathfloatabs@{#2}%
        \expandafter\pgfmathfloatlessthan@\expandafter{\pgfmathresult}{\pgfutilsolvetwotwoleqfloat@to@pgf@range@max}%
        \ifpgfmathfloatcomparison
            % ok.
            \pgfmathfloattofixed{#1}\let\tmpa=\pgfmathresult
            \pgfmathfloattofixed{#2}%
            \edef\pgfmathresult{{\tmpa}{\pgfmathresult}}%
        \else
            % singular (because PGF cannot represent its result
            \let\pgfmathresult=\pgfutil@empty
        \fi
    \else
        % singular (because PGF cannot represent its result
        \let\pgfmathresult=\pgfutil@empty
    \fi
}%

% there are funny programs which overwrite \read and \write (like
% tabularx).
\let\pgfutil@write=\write
\let\pgfutil@read=\read

% curtesy for latex:
\let\pgfutil@protect\relax

% Used by latex for doing reruns
% #1 = an internal label
% #2 = a to-be-tested text
\def\pgfutil@check@rerun#1#2{}

% new tempdims:
\newdimen\pgfutil@tempdima
\newdimen\pgfutil@tempdimb


% luatex stuff
% We assume luatex version > 0.39:
%   - \directlua <general text> will work
%   - \directlua is the only luatex primitive that we can assume
%     accessible without being prefixed by the format via
%     tex.enableprimitives.
% Ideas taken from the ifluatex package (Heiko Oberdiek)
\let\pgfutil@ifluatex\iffalse
\begingroup\expandafter\expandafter\expandafter\endgroup
\expandafter\ifx\csname directlua\endcsname\relax
\else
  \expandafter\let\csname pgfutil@ifluatex\expandafter\endcsname
    \csname iftrue\endcsname
\fi

\pgfutil@ifluatex
  \let\pgfutil@directlua\directlua
  \pgfutil@directlua{%
    tex.enableprimitives('pgfutil@',{'luaescapestring'})}
\else
  \def\pgfutil@directlua#1{}
  \def\pgfutil@luaescapestring#1{}
\fi


\def\pgfutil@shellescape@lua@eightseven#1{%
    \pgfutil@directlua{os.execute("\pgfutil@luaescapestring{#1}")}%
}%
\def\pgfutil@shellescape#1{%
    \immediate\write18{#1}%
}%
\pgfutil@IfUndefined{directlua}{%
}{%
    \pgfutil@IfUndefined{lastsavedimageresourcepages}{%
        % Ah - an old version of luatex. It still supports \write18
    }{%
        % take the new variant...
        \let\pgfutil@shellescape=\pgfutil@shellescape@lua@eightseven
    }%
}%

% End of luatex stuff

% Advances a number stored in a macro and writes the result back into
% the macro.
% #1 is a macro containing a number.
\def\pgfutil@advancestringcounter#1{%
    \begingroup
        \c@pgf@counta=#1\relax
        \advance\c@pgf@counta by1
        \edef#1{\the\c@pgf@counta}%
        \pgfmath@smuggleone#1%
    \endgroup
}%

\endinput
